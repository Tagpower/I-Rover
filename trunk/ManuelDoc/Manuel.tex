\documentclass[a4paper 12pts]{article}
\usepackage[utf8]{inputenc}
\usepackage[T1]{fontenc}
\usepackage[francais]{babel}

\title{Manuel du projet irover}
\date{Lundi 9 Novembre 2015}

\author{}

\begin{document}

\maketitle

Manuel du project irover.


\newpage

\section{Introduction}



Le projet a pour objectif la création d'une application graphique où un petit petit personnage (robot) réaliserait une tache particulière.
Pour cela nous avons imaginé un scénario de mini jeu en deux dimention où notre robot prendrait la forme d'un héros sans peur et sans reproche.
Le manuel suivant se composera de plusieurs partie :
La premiere vous aidera à comprendre le but du jeu ses règles et l'histoire de notre petit monde.
La seconde décrira comment installer l'application et les différentes options d'installation possible. 
La troisième partie sera dédié à la description détaillés de l'infrastructure logicielle sur la forme d'une documentation logicielle. 
et enfin nous vous proposerons un tutoriel retrassant une simulation que nous avons effectué.


\section{iRover, l'histoire d'un héros}


Dans la vallé perdu du Vertou, là où nul aventurier n'a plus mit les pieds depuis des années. On raconte qu'il y a d'ici
plusieurs siècle, un grand pirate du nom de Stevy J. y aurait caché un trésors :"le chamalo magique".
Notre héros, jeune et intrépide aventurier du nom de Rover est à la recherche de cet artefact et se voit donc lancer 
dans cette grande aventure afin de retrouver cette sucrerie antique.
Cependant, la vallé est habité par d'étranges créatures, des chou-kêtes. Cette horde de créature est farouchement attaché à leurs territoire
et n'hésiteront pas à attaquer quiconque qui y posera le pied.

But du jeu :

L'objectif  principal du jeu (ou plutot de notre jeune amis) est de ramasser tous les trésors présents 
sur le terrain afin d'atteindre gloire et richesse et surtout un jour trouver le chamalo magique.

Ces coffres seront disséminés sur la carte, souvent derrière des obstacles que le héros devra contourner. 
Des ennemis (les chou-kêtes) pourront également attaquer notre héros.

Le jeu prendra fin une fois tous les coffres ramassés ou si notre héros ne peut plus continuer son aventure.


\subsection{Gestion de notre héros}

Rover, notre héros, est controlé par une intelligence artificiel d'où son petit surnom de robot.
Il pourra se déplacer sur la carte case par case et rencontrera des obstacles et pourra interagir avec d'autre élément du monde.

N'ayant pas appris à nagé, il ne pourra pas se déplacer sur l'eau et ayant un vertige incontrolabe, il aura du mal à grimper sur les rochers,
les arbres et même les buissons !

Il pourra cependant ramasser des objets au sol, 

\subsection{Gestion des clés}

%Aucune clef sur le terrain pour l'instant

Un certain nombre de clé seront présents sur la carte ou peuvent être obtenus en tuant des ennemies.
chaque ouverture de 
Une fois le coffre ouvert, la clef est perdue et le robot devra en récupérer d'autres pour ouvrir d'autres coffres.

Pour récupérer une clef, il faut que le robot se trouve sur la même case qu'elle. Cependant, un maximum de deux clefs portées par le robot est autorisé. Donc si le robot possède déjà deux clef et qu'il se trouve sur la même case qu'une clef, il la laisse à sa place et ne pourra la prendre uniquement s'il ouvre un coffre entre temps.

\subsection{Gestion des ennemis}
Certains ennemis seront présent sur le terrain qui sont là pour mettre le robot en difficulté. Le robot devra combattre ces ennemis s'il les rencontrent pendant le jeu.

Lorsque le robot se trouve à côté d'un ennemi, il est obligé de le combattre. Celà se traduit par un algorithme de combat qui fera gagner le duel au robot ou à l'ennemi, ce qui entrainera la fin du jeu. Une fois l'ennemi vaincu, celui-ci disparait du terrain.

\subsection{Gestion de l'environnement}
Le terrain de jeu dans lequel le robot devra jouer possède des obstacles. Ceux-ci sont infranchissables. Dans le cas où le robot se trouve face à un obstacle (mur, arbre, cours d'eau\ldots) celui ci devra trouver un moyen de le contourner. Il peut donc tourner à droite, à gauche ou encore faire demi-tour.


\section{Fonctionnement}

\subsection{Les déplacements du robot}
Le robot se déplace uniquement horizontalement et verticalement, il ne peut donc pas se déplacer en diagonale. Il ne pourra se déplacer que d'une case en une case.

Parfois le robot aura le choix du déplacement, dans ce cas il prendra la meilleure solution selon ce qu'il connait du terrain. S'il n'y a qu'un seul chemin possible, le robot devra l'emprunter et le poursuivre jusqu'au bout jusqu'à ce qu'il y ait plusieurs chemins possibles.

Le robot se sert d'un algorithme de recherche pour avancer de la façon la plus optimale possible. Il peut voir les parties de la carte qui se trouvent autour de lui. Lorsque le robot avance, il découvre donc une nouvelle partie de la carte et conserve également la carte déjà parcourue.

\subsection{L'environnement}
La carte sur laquelle le robot se déplace est construite à l'aide de tiled. La carte modélisée, on distingue deux types de cases:

\begin{enumerate}
	\item les cases accessibles par le robot
	\item les cases qui bloquent le robot
\end{enumerate}

La carte est sous forme d'un fichier XML que le programme interprête pour renvoyer une carte exploitable par le robot.


\end{document}


