\documentclass[a4paper 12pts]{article}
\usepackage[utf8]{inputenc}
\usepackage[T1]{fontenc}
\usepackage[francais]{babel}
\usepackage{graphicx}


\title{Rapport de Projet : iRover}

%mettez vos noms svp !!

\author{R. Joachim CLAYTON}
\author{Geoffrey DESBROSSES}
\author{Clément BAUCHET}

\begin{document}

\maketitle


\begin{figure}[h]
   \includegraphics[width=350pt]{Illustration/proj_irover.jpg}
	\caption{iRover, l'histoire d'un héros qu'on appellait robot}
\end{figure}



\newpage


\renewcommand{\contentsname}{Sommaire} 
\tableofcontents

\newpage








\section{Rapport de Projet}


\vspace{2cm}



\subsection{Introduction}
\subsection{Membre du groupe et tâches de chacun}



\vspace{1 cm}

\begin{tabular}{| c | p{6cm} | c |}
\hline
\textbf{Responsable} & \textbf{Tâche} & \textbf {Description} \\
\hline
Clément Bauchet & Création graphique & La création de l'environnement graphiques et des images \\					
& Parsing de la carte & L'interprétation entre les objets graphiques et les objets C++\\					
\hline				
Geoffrey Desbrosses & Création des objets & Création de chaque objet de l'application et de leurs actions\\
& Documentation du code source & Gestion de la documentation Doxygen \\ 
\hline
& IHM & Interface utilisateur de l'application. \\
\hline
Jean-Christophe Guerin & IA  & Définitions des algorithmes de pathfinding de déplacement et de découverte\\	
\hline
Joachim Clayton & Documentation utilisateur & Redaction de la documentation utilisateur \\                 
\end{tabular}

\section{Définition de la carte}

Pour commencer, il fallait générer un ou plusieurs terrains à l'aide du logiciel \emph{Tiled}, un éditeur de cartes. La carte est créée avec une certaine tilesheet (une image contenant le dessin de toutes les cases possibles), et enregistrée sous le format \textbf{.tmx}, un format de type XML.

Pour pouvoir afficher le terrain dans notre application, nous avons d'abord fait appel au module \emph{ClanXML} de la bibliothèque \emph{ClanLib} qui est utilisée pour notre application, notamment pour l'affichage. Une classe \textbf{Map} a été créée, représentant le terrain et ses informations : les dimensions, les types de ses cases et la praticabilité de celles-ci par le robot ou d'autres entités. A la création d'une carte, ces informations sont récupérées directement depuis le fichier de carte .tmx fourni, en le parsant en tant que document DOM comme permis par \emph{ClanXML}. \\
\emph{N.B. :} A l'heure actuelle, il n'est pas possible de déterminer par seule lecture du fichier .tmx quels types de cases sont praticables par le robot. Ceci est déterminé au cas par cas dans le code. De plus, la praticabilité des cases se résume simplement à la possibilité ou non d'aller dessus. Nous pourrions définir un coût de déplacement pour chaque type de case, qui serait alors pris en compte dans l'évaluation du chemin optimal du robot.

Une fois les données de la carte obtenues, il nous faut afficher cette dernière dans la fenêtre principale de l'application. Comme nous ne disposons que de données brutes sur les cases (un nombre indiquant quelle texture sera utilisée pour dessiner la case), nous avons créé une autre classe, \textbf{Tileset}, qui représente l'image regroupant toutes les textures utilisables pour dessiner les cases (la \emph{tilesheet} décrite précédemment), ainsi que ses informations : ses dimensions, la taille des cases en pixels et l'espace entre chaque texture dans l'image. Là encore, ces données peuvent être retrouvées directement depuis le même fichier \emph{.tmx} utilisé pour créer la carte.
En connaissant le numéro d'une case de la carte, on peut alors retrouver quelle texture utiliser pour la dessiner. Il est donc désormais possible d'afficher toute la carte.

Enfin, nous avons défini la classe \textbf{mapDisplay}, qui sert actuellement de classe principale pour l'exécution de l'application. C'est ici que toutes les entités intervenant dans le déroulement du jeu sont affichées : la carte, le robot, ses ennemis et les objets.

Une possibilté d'extension serait de permettre au robot d'interagir directement sur son environnement et modifier le terrain au fur et à mesure de son exploration (\emph{e.g.} en construisant des ponts, ou détruisant des murs...), ce que la structure de Map permet car la carte n'est pas en lecture seule après l'analyse du fichier \emph{.tmx}. 


%partie Geoffrey

\section{Définition des objets}
La définition des objets comprend la création de tous les objets Personnage, Coffre, Clef, Arme\ldots et toutes les actions entre ces objets.

\subsection{Les objets existants}
Le robot possède une fonction pour se déplacer et modifier ses coordonnées. Il possède également une méthode combattre propre à notre sujet. Des actuateurs arme et armure ont été implémentés. Le robot peut intéragir avec d'autres objets tels que les ennemis, les clefs ou les coffres. Deux types d'armes ont également été créé (bazooka et mine) et les classes Arme et Armures permettent d'implémenter facilement de nouvelles armes et armures ayant leurs propres caractéristiques.

\subsection{Les objets manquant}
Le robot ne possède pas de senseur. L'idée était d'implémenter un scanner qui pouvait scanner une carte et savoir quels étaient les cases du terrain où le robot pouvait aller. Aucun type d'armure n'a été implémenté, cependant les personnages peuvent tout de même avoir une armure par défaut avec une robustesse et un nom. Ainsi ça ne modifie pas le comportement des personnages.

\subsection{gestion des évênements}
Certains objets peuvent intéragir entre eux, par exemple la mine peut exploser au contact d'un ennemi, deux personnages peuvent s'affronter, le robot peut ramasser des clefs et ouvrir des coffres. Cependant, ces actions ne sont pas visibles sur l'interface graphique, excepté le déplacement du robot.

\subsection{Le robot}
\subsection{Les ennemis}
\subsection{Les coffres}
\subsection{Les clés}
\subsection{La gestion des évènements}
\subsubsection {Rencontre avec un ennemi} 
\subsubsection {Ouverture d'un coffre}
\subsubsection {Ramasser une clé}

\section{IHM}



\section{IA}

%partie JC

\subsection{path finding}
\subsection{découverte de la carte}



\section{Problèmes rencontrés}

Malgré un début difficile en raison de problèmes à la compilation et l'édition des liens de ClanLib, cette bibliothèque s'est avérée assez facile et intuitive à utiliser, ce qui a permis de vite obtenir un affichage de l'application.

\section{Conclusion}



\end{document}

