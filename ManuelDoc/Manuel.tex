\documentclass[a4paper 12pts]{article}
\usepackage[utf8]{inputenc}
\usepackage[T1]{fontenc}
\usepackage[francais]{babel}

\title{Manuel du projet irover}
\date{Lundi 9 Novembre 2015}

\author{}

\begin{document}

\maketitle

Manuel du project irover.

%mettre une image jolie de jeu rpg super cute wayyyyy

\newpage

\tableofcontents

\section{Introduction}

%Intro


Le projet a pour objectif la création d'une application graphique où un petit personnage (robot) réaliserait une tache particulière.
Pour cela nous avons imaginé un scénario de mini jeu semblable au jeux en deux dimensions où notre robot prendrait la forme d'un héros sans peur et sans reproche.
Le manuel suivant se composera de plusieurs parties :
La premiere vous aidera à comprendre le but du jeu ses règles et l'histoire de notre petit monde.
La seconde décrira comment installer l'application et les différentes options d'installation.
La troisième partie constitura la documentation utilisateur.
La quatrième quand a elle sera dédié à la description détaillés de l'infrastructure logicielle sur la forme d'une documentation dédier aux développeurs ou utilisateurs confirmé et enfin nous vous proposerons un tutoriel retrassant une simulation pour illustré le rendement de notre jeu.


\section{iRover, l'histoire d'un héros}

%background, histoire, scénario


Dans la vallé perdu du Vertou, là où nul aventurier n'a plus mit les pieds depuis des années. On raconte qu'il y a 
plusieurs siècle, un grand pirate du nom de Stevy J. y aurait caché un trésors :"le chamalo magique".
Notre héros, jeune et intrépide aventurier du nom de Rover est à la recherche de cet artefact et se voit donc lancer 
dans cette grande quete afin de retrouver cette sucrerie antique.
Cependant, la cette mistérieuse vallé est habité par d'étranges créatures, des chou-kêtes. 
Cette horde de créature est farouchement attaché à leur territoire et n'hésiteront pas à attaquer quiconque y pénetrera.


%regle du jeu

But du jeu :

L'objectif  principal du jeu (ou plutot de notre jeune amis) est de ramasser tous les trésors présents 
sur le terrain afin d'atteindre gloire et richesse et surtout un jour trouver le chamalo magique.

Ces coffres seront disséminés sur la carte, souvent derrière des obstacles que le héros devra contourner. 
Des ennemis (les chou-kêtes) pourront également attaquer notre héros et donc un combat sans merci s'enclanchera entre héros et monstre.

Le jeu prendra fin une fois tous les coffres ramassés ou si notre héros ne peut plus continuer son aventure.


\subsection{Rover, un héros pas comme les autres}



Rover, notre héros, est controlé par une intelligence artificiel d'où son petit surnom de robot.
Il pourra se déplacer sur la carte, case par case et rencontrera des obstacles. 
Il pourra interagir avec d'autre éléments du monde. 

N'ayant pas appris à nagé et du à des crises de vertiges récurente, il ne pourra ni se déplacer sur l'eau ni grimper sur les rochers,
les arbres, les murs et même les buissons !

Il pourra cependant ramasser des objets au sol (des clefs), se battre contre les ennemis et ouvrir les coffres. 

\subsection{Les iClef}

%Aucune clef sur le terrain pour l'instant

Un certain nombre de clé seront présents sur la carte ou peuvent être obtenus en tuant des ennemies.
Pour certains coffre, il peut etre necessaire d'avoir une clef. 
Une fois le coffre ouvert, la clef est consomé et le robot devra en récupérer d'autres pour ouvrir d'autres coffres.

Pour récupérer une clef, il faut que le robot se trouve sur la même case que celle ci, elle disparait alors de la carte et 
viens s'ajouter à son inventaire de clef.

\subsection{Les Chou-kêtes}

Certains ennemis seront présent sur le terrain et mettrons le robot en difficulté. 
Le robot devra alors combattre ces ennemis s'il les rencontre afin de rester en vie.

Lorsque le robot se trouve à côté d'un ennemi, il est obligé de le combattre. 
Celà se traduit par un algorithme de combat qui fera gagner le duel au robot ou à l'ennemi. 

Une fois l'ennemi vaincu, celui-ci disparait du terrain mais dans le cas contraire, 
notre héros ne sera plus en mesure de continuer et notre jeu prendra fin.



\section{Manuel d'installation}

\subsection{makefile}

%expliquer l'installation

%expliquer les librairies à utiliser
%parler de l'option pour que ça compile avec tout les systemes d'exploitation

\subsubsection{gestion des systeme d'exploitation}

\subsection{.config ????????}
%a virer si on utilise pas
%les options du makefile
%a ambélir si on a des idées (rappelle que meme si c'est pas implémenté, ça peut nous donner des points)


\section{Documentation utilisateur}

%documentation utilisateur : donner des informations à des utilisateurs lambda sans grande connaissance informatique, avec des mots simple sur le fonctionnement de l'application.
%donner egalement une description simple de comment est faite ou est géré les partie de l'application.

\subsection{Les personnages}
Le jeu possède deux type de personnages: le robot et les ennemis. Une classe nommée Personnage permet de réaliser les actions de ces deux types de personnages, soit:
\begin{enumerate}
	\item se déplacer
	\item combattre
\end{enumerate}
Chaque personnage possède des attributs positionX et positionY qui sont respectivements les position du personnage en X et en Y.

\subsubsection{Le robot}
Le robot se déplace uniquement horizontalement et verticalement, il ne peut donc pas se déplacer en diagonale. 
Il ne pourra se déplacer que d'une case à la foi.


Le robot, en plus de pouvoir se déplacer sur le terrain, peut aussi ramasser des clefs et ouvrir des coffres.
Pour celà, le robot possède un entier nommé "inventaire" qui représente le nombre de clefs que possède le robot à un instant.
Lorsque la fonction ramasser(Clef* clef) est appelée, l'inventaire du robot va s'incrémenter de 1 et lorsque la méthode ouvrir(Coffre* coffre)
est appelée, l'inventaire va se décrémenter de 1.



\subsubsection{les ennemis}

Les ennemis ont les mêmes propriétés que la classe Personnage, c'est juste leur manière de se déplacer qui change par rapport au robot. 
Un ennemi peut se déplacer sur toute la carte de manière aléatoire grâce à un algorithme qui génère un déplacement aléatoire.
Lorsqu'un ennemi est proche du robot, un combat se lance entre les deux personnages.

\subsection{les coffres}

\subsection{les clé}

\subsection{L'environnement}
La carte sur laquelle le robot se déplace est construite à l'aide de Tiled, il s'agit d'un logicielle permettant de créer des cartes case par case. 
Pour les cartes modélisées, on distingue deux types de cases:

L'environnement comprend la gestion des coffres et des clefs.
Une clef possède des attributs positionX et positionY qui sont ses positions en X et en Y sur la carte. 
Un coffre possède les mêmes attributs mais un statut en plus qui est simplement une valeur booléenne qui dit si le coffre est ouvert ou fermé.


\begin{enumerate}
	\item les cases ou le robot peut se déplacer
	\item les cases ou le robot ne peut pas se déplacer
\end{enumerate}

La carte est sous forme d'un fichier XML que le programme interprête pour renvoyer une carte exploitable par le robot.

L'environement servira de base également pour la disposition des autres entités présente, coffre, clef ennemie.
Ceux ci sont disposé aléatoirement sur la carte à des endroits accessibles par le héros mais ils ne peuvent pas se déplacer.
La rencontre avec un de ces éléments entraine la gestion des évênements.





\subsection{La gestion des évênements}

Le robot, notre amis rover, devra faire face à beaucoup d'évênements, ouvrir un coffre, se battre contre un ennemi, ramasser une clef, se cogner contre un mur etc..
Tout ceci est géré par le comportement de chaque objets mais aussi pour sa hierarchi sur la carte.

\subsubsection {Rencontre avec un ennemi} 
Un ennemis et notre héros devront avoir la même importance physique : il ne peuvent pas se supperposer.
Un ennemis et notre héros n'auront besoin que d'un regard pour enclanché le combat ! 1 case de différence.

%mettre un schéma

\subsubsection {Ouvertude d'un coffre}
Un coffre et notre héros n'ont pas la même importance physique : le héros devra pouvoir "marcher" sur le coffre.
Un coffre et notre héros devront impérativement être sur la même case pour que le coffre puisse s'ouvrir et ce dernier, 
le héros, devra avoir une clé pour que cela soit possible.

%mettre un schéma

\subsubsection {Ramasser une clé}
Une clé et notre héros n'ont pas la même importance physique : le héros devra pouvoir "marcher" sur la clé.

%mettre un schéma
 
\subsection{l'interface utilisateur}

\subsection{IA}

\subsubsection{path finding}

\subsubsection{découvert de la carte}

\subsection{condition fin de jeu}


















\section{tutoriel}

\end{document}


