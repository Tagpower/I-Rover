\documentclass[a4paper 12pts]{article}
\usepackage[utf8]{inputenc}
\usepackage[T1]{fontenc}
\usepackage[francais]{babel}

\title{Manuel du projet "faut trouver un nom"}
\date{Lundi 9 Novembre 2015}
\author{}

\begin{document}

\maketitle

\section{Introduction}

Le projet a pour objectif la création d'un robot autonome réalisant une tâche particulière.
Pour cela nous avons imaginé un scénario sous forme de mini jeu dont le personnage principal est notre robot.
Le manuel suivant vous aidera à comprendre le but du jeu et les règle de celui ci et vous proposeras une description détaillés de son infrastructure logicielle réserver aux developpeur et utilisateurs confirmé mais aussi une section tutoriel qui reprendra un exemple d'utilisation de notre petite application.

(#TODO On doit segmenter le manuel en plusieur partie, une partie destiné à tout le monde avec les regles du jeux etc, une un peu plus technique ou on parlera du code et des differente techno, et une autre ou on décrit ce qui se passe.
On pourra aussi imaginé des scénario ou des options possible pour permettre de meublé ou donné plus de contenu à notre projet meme si ce sont des choses qu'on ne fait pas concraitement)

\section{But du jeu}


L'objectif du jeu est de ramasser tous les trésors présents sur le terrain.

%Pour le moment : 1 coffre (augmentation de la difficulter au fur et à mesure)

Cependant certains coffres seront cachés derrière des obstacles que le robot devra trouver. Des ennemis seront également présent dans le jeu afin d'ajouter une difficulté supplémentaire au robot.
Une fois tous les coffres ramassés, le jeu s'arrete.
Si le robot meurt au combat, le jeu s'arrête également.


\subsection{Gestion des coffres}

%Pour le moment, le coffre s''ouvre sans clef

Un certain nombre de coffres seront présents ceux-ci ne peuvent s'ouvrir qu'à l'aide d'une clef préalablement obtenue.
Le robot ne peut ouvrir un coffre que s'il possède une clef et qu'il se trouve sur la même case que le coffre.

%Le coffre possedera deux sprites, un ouvert un fermer.

\subsection{Gestion des clés}

%Aucune clef sur le terrain pour l'instant

Un certain nombre de clé seront présents sur la carte ou peuvent être obtenus en tuant des ennemies.
chaque ouverture de 
Une fois le coffre ouvert, la clef est perdue et le robot devra en récupérer d'autres pour ouvrir d'autres coffres.

Pour récupérer une clef, il faut que le robot se trouve sur la même case qu'elle. Cependant, un maximum de deux clefs portées par le robot est autorisé. Donc si le robot possède déjà deux clef et qu'il se trouve sur la même case qu'une clef, il la laisse à sa place et ne pourra la prendre uniquement s'il ouvre un coffre entre temps.

\subsection{Gestion des ennemis}
Certains ennemis seront présent sur le terrain qui sont là pour mettre le robot en difficulté. Le robot devra combattre ces ennemis s'il les rencontrent pendant le jeu.

Lorsque le robot se trouve à côté d'un ennemi, il est obligé de le combattre. Celà se traduit par un algorithme de combat qui fera gagner le duel au robot ou à l'ennemi, ce qui entrainera la fin du jeu. Une fois l'ennemi vaincu, celui-ci disparait du terrain.

\subsection{Gestion de l'environnement}
Le terrain de jeu dans lequel le robot devra jouer possède des obstacles. Ceux-ci sont infranchissables. Dans le cas où le robot se trouve face à un obstacle (mur, arbre, cours d'eau\ldots) celui ci devra trouver un moyen de le contourner. Il peut donc tourner à droite, à gauche ou encore faire demi-tour.


\section{Fonctionnement}

\subsection{Les déplacements du robot}
Le robot se déplace uniquement horizontalement et verticalement, il ne peut donc pas se déplacer en diagonale. Il ne pourra se déplacer que d'une case en une case.

Parfois le robot aura le choix du déplacement, dans ce cas il prendra la meilleure solution selon ce qu'il connait du terrain. S'il n'y a qu'un seul chemin possible, le robot devra l'emprunter et le poursuivre jusqu'au bout jusqu'à ce qu'il y ait plusieurs chemins possibles.


\end{document}


